\chapter{Theory and Fundamentals - Nanostructured Materials, Charge, Zeta Potential, Aqueous Kinetics}

\label{chap-three}

\section{Nano-porous Alumina}

Experiments conducted on a QCM surface benefit from an increase of surface area. As the surface area increases, in general the amount of mass adsorbed and energy dissipated, all other environmental and material factors being equal, will increase. In effect, increasing the surface area acts as a gain. To that end, many aluminum samples in this work have been anodized unto the formation of pores.(cite Toborek, etc)

In an Oxalic acid bath, the Aluminum-coated QCM is the cathode.  Applying a voltage to the system, charges on the surface of the cathode self-organize in a hexagonal close-packed pattern and begin the oxidation process, boring into the aluminum substrate.




\subsection{Electronic Surface Structure of Alumina}

\subsection{Geometry of Nano-porous Alumina and Effects on Phase Transitions}





\section{Graphene Membranes}

\subsection{Electronic Surface Structure of Nanodiamonds}






\section{Graphene Oxide Membranes}

\subsection{Electronic Surface Structure of Nanodiamonds}







\section{Nanodiamonds}

\subsection{Detonation Nanodiamond Synthesis}



\subsection{Molecular Adsorption and Bonding of Functional Groups to Nanodiamond Surfaces}

\subsubsection{Electronic Surface Structure of Nanodiamonds}



\subsection{Functional Groups, Surface Charge, and Zeta Potential}



\subsection{Aqueous Kinetics of Nanodiamonds}



\addcontentsline{toc}{section}{{Chapter 3 References}}

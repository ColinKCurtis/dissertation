%% ------------------------------ Abstract ---------------------------------- %%
\begin{abstract}

Driven by environmental and energy use reduction concerns, interest in and funding for research into advances in lubrication and filtration continues at an accelerating pace. Under the supervision of Dr. Jacqueline Krim, my tribological research has utilized robust nano-structured materials in pursuit of improved scientific understanding of interface mechanics. Nanoparticle lubrication studies carry the potential for reduction of energy consumption and machine wear. Characterization of nano-scale carbon membranes capable of selectively impeding or blocking gas permeability gives the opportunity to improve filtration or separation processes. In pursuit of these practical goals, our work focused on understanding the fundamental physical features which govern the tribological interactions between nanoscale-structured materials and solids, liquids, or gases in contact with them.

In my nanotribological studies, the Quartz Crystal Microbalance (QCM) has been the primary tool for investigation of dissipation and wear of alumina, stainless steel, and carbon surfaces in the presence of water, ethanol, and nanoparticle solutions. Additional critical instruments included an Atomic Force Microscope (AFM), Macroscale Tribometer (MTM), Scanning Electron Microscope (SEM), and ZetaSizer (ZS). The tribological properties of these materials, employed throughout academic research and industrial applications, are of great interest from both a fundamental and applied perspective. Determination of their nanometer-scale properties will have lasting impact for issues of the environment, medicine, and national security. The main purpose of this thesis is to describe the critical nanometer-scale features, including geometry, chemical and electronic structures, and how those function to influence tribology on both the nanoscopic and macroscopic length scales.

The first studies which I conducted were of mass uptake and energy dissipation of gaseos ethanol and water vapor through hydrogenated graphene (HG) and graphene oxide flake (HGO) membranes grown on QCM-mounted porous-alumina substrates. The pores, 20nm in diameter and 10um deep, provided the ability to sense diffusion of cubic milimeters a gas species and with a surface area for adsorption greater than 20 square centimeters on a QCM surface with cross sectional area of less than one half of one centimeter square. Motivated by Geim et al.’s 2012 finding that water could permeate through a sub-micrometer-thick GO membrane while helium could not. Our work showed that HGO displayed surface adsorption increases and only minor penetration decreases for both water and ethanol. The HGR membrane, however, shows a distinct, reversible, and repeatable decrease in its permeability to water but only a minor decrease in its permeability to ethanol (surface adsorption also increase for the HGR sample but not as substantially as for the GO to HGO case).

Following the G/GO membrane work, and making up the largest portion of my tribological research, were studies pertaining to exposore of either stainless steel, alumina, or gold substrates to aqueous nanodiamond solutions. Well-regarded for their ability to lubricate and prevent wear in oil-based media, our work focused on aqueous solutions of detonation nano-diamonds coated in either hydroxl or carboxyl groups. Behavior at a solid-liquid interface was examined using a small-volume flow-cell. QCM data, collected before, during, and after introduction and exposure of the three aforementioned substrates to nanodiamond solutions, provided critical information about the nanoscale interactions between particle and substrate. AFM studies of the QCMs before and after exposure revealed details about the changes in surface roughness while SEM studies allowed identification of nanoparticle adhesion, or lack thereof. Finally, macroscale tribometer tests using stainless steel-on-stainless steel as well as alumina-on-alumina interfaces provided friction force information for correlation with known and investigated bulk, surface, and nanoscale material properties for the constituent materials of the experiments.

Collectively, these nanoscale and macroscale tribological studies demonstrate the impact of nano-scale features, include geometry and chemical functionalization, on friction and dissipation at the macroscale. By utilizing several techniques and correlating patterns between them, we have furthered the understanding of the electro-chemical impact of nanometer-size features and chemical functionalization on the tribology of solid-solid and solid-liquid interfaces in the context of carbon, stainless, and alumina.

\end{abstract}


%% ---------------------------- Copyright page ------------------------------ %%
%% Comment the next line if you don't want the copyright page included.


%% -------------------------------- Title page ------------------------------ %%
\maketitlepage

%% -------------------------------- Dedication ------------------------------ %%
\begin{dedication}
 \centering To my wife, Sanjana Curtis, the truest scientist I have known.
 
 And, to my brother, Andrew Curtis, whose capacity is out-stripped by his curiosity. 
 
 
 'Wir m\"ussen wissen — wir werden wissen.' Hilbert, 1930
\end{dedication}

%% -------------------------------- Biography ------------------------------- %%
\begin{biography}
The author was born in Niagara Falls, New York. He lived less than a kilometer from the Niagara River for his childhood. The waters are a deep blue-green and beautiful. \ldots
\end{biography}

%% ----------------------------- Acknowledgements --------------------------- %%
\begin{acknowledgements}
Thank you Dr. Jacqueline Krim, your excitement and cutting intellect kept me on my toes and your patience kept me in my position. Thank you Dr. Harald Ade, for presenting the opportunity to be at NCSU Physics and engaging with me on many levels. Thank you to Dr. Alex Smirnov, for his continuous stream of thoughtful questions, woven through with humor. Thank you to Dr. John Thomas, for his humility and willingness to work with me. I am grateful to Thank you to my wife, for re-kindling my interest in knowing for the sake of knowing.
\end{acknowledgements}

\thesistableofcontents



\thesislistoftables

\thesislistoffigures
